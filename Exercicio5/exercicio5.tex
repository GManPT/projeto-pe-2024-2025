\documentclass[11pt,a4paper]{article}

% Packages for formatting and graphics
\usepackage[utf8]{inputenc}
\usepackage[portuguese]{babel}
\usepackage{geometry}
\usepackage{graphicx}
\usepackage{listings}
\usepackage{xcolor}
\usepackage{amsmath}
\usepackage{amsfonts}
\usepackage{amssymb}
\usepackage{tcolorbox}

% Page geometry
\geometry{margin=2.5cm}

% R code styling
\definecolor{codegreen}{rgb}{0,0.6,0}
\definecolor{codegray}{rgb}{0.5,0.5,0.5}
\definecolor{codepurple}{rgb}{0.58,0,0.82}
\definecolor{backcolour}{rgb}{0.95,0.95,0.92}

\lstdefinestyle{rstyle}{
    backgroundcolor=\color{backcolour},   
    commentstyle=\color{codegreen},
    keywordstyle=\color{magenta},
    numberstyle=\tiny\color{codegray},
    stringstyle=\color{codepurple},
    basicstyle=\ttfamily\footnotesize,
    breakatwhitespace=false,         
    breaklines=true,                 
    captionpos=b,                    
    keepspaces=true,                 
    numbers=none,                    
    showspaces=false,                
    showstringspaces=false,
    showtabs=false,                  
    tabsize=2,
    frame=single,
    rulecolor=\color{blue!30!black}
}

\lstset{style=rstyle}

\begin{document}

\begin{tcolorbox}[colback=blue!5!white,colframe=blue!75!black,title=Jogo do Grão-Duque da Toscana - Exercício 5]

Conta-se que o Grão-Duque da Toscana (séc. XVI) jogava assiduamente com um adversário o seguinte jogo:

\begin{itemize}
    \item Em cada jogada, um dos jogadores (sejam A e B) lança três dados cúbicos perfeitos e soma as pontuações obtidas. Se a soma das pontuações for 9 (denotado por ``soma 9''), o jogador A ganha a jogada e, se a soma das pontuações for 10 (denotado por ``soma 10''), será o jogador B a ganhar; caso contrário, ninguém ganha.
\end{itemize}

O que intrigava o Grão-Duque era que, apesar de 9 e 10 pontos se poderem decompor cada um em 6 maneiras diferentes, o jogador A ganhava com menor frequência do que o jogador B.

Fixando a semente em 2318, simule \( n = 38000 \) jogadas desse jogo. Reporte o valor da diferença entre as frequências relativas com que a ``soma 10'' e a ``soma 9'' são obtidas nessas \( n \) jogadas, arredondada a 4 casas decimais.

\textbf{Nota:} caso use a função \texttt{sample} na simulação dos lançamentos dos dados, NÃO deverá especificar as probabilidades dos resultados possíveis. Por omissão, estes serão considerados equiprováveis.

\end{tcolorbox}

\section*{Código R}

\begin{lstlisting}[language=R]
# Set seed for reproducibility
set.seed(2318)

# Number of games to simulate
n <- 38000

# Initialize counters
sum_9_count <- 0
sum_10_count <- 0

# Simulate n games
for (i in 1:n) {
  # Roll three dice
  dice1 <- sample(1:6, 1)
  dice2 <- sample(1:6, 1)
  dice3 <- sample(1:6, 1)
  
  # Calculate sum
  total_sum <- dice1 + dice2 + dice3
  
  # Count wins for each player
  if (total_sum == 9) {
    sum_9_count <- sum_9_count + 1
  } else if (total_sum == 10) {
    sum_10_count <- sum_10_count + 1
  }
}

# Calculate relative frequencies
freq_sum_9 <- sum_9_count / n
freq_sum_10 <- sum_10_count / n

# Calculate difference
difference <- freq_sum_10 - freq_sum_9

# Display results
cat("Numero de jogadas simuladas:", n, "\n")
cat("Numero de 'soma 9':", sum_9_count, "\n")
cat("Numero de 'soma 10':", sum_10_count, "\n")
cat("Frequencia relativa 'soma 9':", freq_sum_9, "\n")
cat("Frequencia relativa 'soma 10':", freq_sum_10, "\n")
cat("Diferenca (soma 10 - soma 9):", difference, "\n")
cat("Diferenca arredondada a 4 casas decimais:", round(difference, 4), "\n")
\end{lstlisting}

\section*{Resultados}

\begin{tcolorbox}[colback=green!5!white,colframe=green!75!black,title=Solução]
Para a simulação de 38000 jogadas do jogo do Grão-Duque da Toscana:

\begin{center}
\begin{tabular}{|l|c|}
\hline
\textbf{Resultado} & \textbf{Valor} \\
\hline
Número de 'soma 9' & 4426 \\
Número de 'soma 10' & 4779 \\
\hline
Frequência relativa 'soma 9' & 0.1164737 \\
Frequência relativa 'soma 10' & 0.1257632 \\
\hline
\textbf{Diferença (soma 10 - soma 9)} & \textbf{0.009289474} \\
\hline
\end{tabular}
\end{center}

\vspace{0.5cm}
\textbf{Resposta final:} A diferença entre as frequências relativas, arredondada a 4 casas decimais, é \boxed{0.0093}.

\vspace{0.3cm}
\textit{Observação:} Como esperado, a frequência relativa da 'soma 10' é maior que a da 'soma 9', confirmando a observação histórica do Grão-Duque.
\end{tcolorbox}

\end{document}
