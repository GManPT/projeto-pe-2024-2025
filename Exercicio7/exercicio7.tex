\documentclass[11pt,a4paper]{article}

% Packages for formatting and graphics
\usepackage[utf8]{inputenc}
\usepackage[portuguese]{babel}
\usepackage{geometry}
\usepackage{graphicx}
\usepackage{listings}
\usepackage{xcolor}
\usepackage{amsmath}
\usepackage{amsfonts}
\usepackage{amssymb}
\usepackage{tcolorbox}

% Page geometry
\geometry{margin=2.5cm}

% R code styling
\definecolor{codegreen}{rgb}{0,0.6,0}
\definecolor{codegray}{rgb}{0.5,0.5,0.5}
\definecolor{codepurple}{rgb}{0.58,0,0.82}
\definecolor{backcolour}{rgb}{0.95,0.95,0.92}

\lstdefinestyle{rstyle}{
    backgroundcolor=\color{backcolour},   
    commentstyle=\color{codegreen},
    keywordstyle=\color{magenta},
    numberstyle=\tiny\color{codegray},
    stringstyle=\color{codepurple},
    basicstyle=\ttfamily\footnotesize,
    breakatwhitespace=false,         
    breaklines=true,                 
    captionpos=b,                    
    keepspaces=true,                 
    numbers=none,                    
    showspaces=false,                
    showstringspaces=false,
    showtabs=false,                  
    tabsize=2,
    frame=single,
    rulecolor=\color{blue!30!black}
}

\lstset{style=rstyle}

\begin{document}

\begin{tcolorbox}[colback=blue!5!white,colframe=blue!75!black,title=Estimação por Máxima Verossimilhança - Exercício 7]

Considere que a variável aleatória \( X \) representa o comprimento (em cm) dos ovos de uma dada espécie de pássaro e que pode ser modelada pela função densidade de probabilidade

\[
f_X(x) = \frac{\lambda^\alpha}{\Gamma(\alpha)} x^{\alpha-1} e^{-\lambda x},
\]

para \( x > 0 \), onde \(\alpha\) e \(\lambda\) são parâmetros com valores positivos desconhecidos e

\[
\Gamma(\alpha) = \int_0^\infty t^{\alpha-1} e^{-t} \, dt
\]

denota a função Gama.

Ao deduzir as estimativas de máxima verossimilhança de \(\alpha\) e \(\lambda\), \(\hat{\alpha}\) e \(\hat{\lambda}\), constatará que \(\hat{\lambda}\) se pode escrever como função de \(\hat{\alpha}\) mas que não existe uma solução explícita para \(\hat{\alpha}\). No entanto, \(\hat{\alpha}\) pode ser obtida numericamente por recurso à função \texttt{uniroot} do R. Use o intervalo \([62.2, 77.8]\) como intervalo inicial de pesquisa e não utilize qualquer outro argumento opcional dessa função.

Assuma que \((X_1, \ldots, X_n)\) é uma amostra aleatória de \(X\) e que a observação de \(n = 16\) ovos dessa espécie de pássaro resultou em \(\sum_{i=1}^n x_i = 120.68\) e \(\sum_{i=1}^n \log x_i = 32.2\).

Determine a estimativa de máxima verossimilhança de \((\alpha - 1)/\lambda\), o comprimento modal dos ovos dessa espécie de pássaro, indicando-a arredondada a 2 casas decimais.

\end{tcolorbox}

\section*{Código R}

\begin{lstlisting}[language=R]
# Given data
n <- 16
sum_x <- 120.68
sum_log_x <- 32.2

# Sample statistics
x_bar <- sum_x / n
log_x_bar <- sum_log_x / n

cat("Dados:\n")
cat("n =", n, "\n")
cat("sum(x_i) =", sum_x, "\n")
cat("sum(log(x_i)) =", sum_log_x, "\n")
cat("x_bar =", x_bar, "\n")
cat("log_x_bar =", log_x_bar, "\n\n")

# For Gamma distribution with pdf f(x) = (lambda^alpha / Gamma(alpha)) * x^(alpha-1) * exp(-lambda*x)
# The MLE equations are:
# 1) lambda_hat = alpha_hat / x_bar
# 2) digamma(alpha_hat) - log(alpha_hat) + log(x_bar) - log_x_bar = 0

# Define the equation to solve for alpha_hat
# digamma(alpha) - log(alpha) + log(x_bar) - log_x_bar = 0
equation_for_alpha <- function(alpha) {
  digamma(alpha) - log(alpha) + log(x_bar) - log_x_bar
}

# Use uniroot to find alpha_hat in the interval [62.2, 77.8]
alpha_hat <- uniroot(equation_for_alpha, interval = c(62.2, 77.8))$root

# Calculate lambda_hat using the relationship lambda_hat = alpha_hat / x_bar
lambda_hat <- alpha_hat / x_bar

# Calculate the modal length: (alpha - 1) / lambda
modal_length <- (alpha_hat - 1) / lambda_hat

cat("Estimativas de Maxima Verossimilhanca:\n")
cat("alpha_hat =", alpha_hat, "\n")
cat("lambda_hat =", lambda_hat, "\n")
cat("Comprimento modal (alpha_hat - 1) / lambda_hat =", modal_length, "\n")
cat("Comprimento modal arredondado a 2 casas decimais =", round(modal_length, 2), "\n")

# Verification: check if our alpha_hat satisfies the equation
cat("\nVerificacao:\n")
cat("Valor da equacao em alpha_hat:", equation_for_alpha(alpha_hat), "\n")
\end{lstlisting}

\section*{Resultados}

\begin{tcolorbox}[colback=green!5!white,colframe=green!75!black,title=Solução]
Para a estimação por máxima verossimilhança da distribuição Gama com os dados observados:

\begin{center}
\begin{tabular}{|l|c|}
\hline
\textbf{Estatística/Estimativa} & \textbf{Valor} \\
\hline
\( n \) & 16 \\
\( \sum_{i=1}^n x_i \) & 120.68 \\
\( \sum_{i=1}^n \log x_i \) & 32.2 \\
\( \bar{x} \) & 7.5425 \\
\( \overline{\log x} \) & 2.0125 \\
\hline
\( \hat{\alpha} \) & 62.24955 \\
\( \hat{\lambda} \) & 8.253173 \\
\hline
\textbf{Comprimento modal} \( \frac{\hat{\alpha} - 1}{\hat{\lambda}} \) & \textbf{7.421334} \\
\hline
\end{tabular}
\end{center}

\vspace{0.5cm}
\textbf{Resposta final:} A estimativa de máxima verossimilhança de \( \frac{(\alpha - 1)}{\lambda} \), arredondada a 2 casas decimais, é \boxed{7.42}.

\vspace{0.3cm}
\textit{Observação:} A verificação confirma que \( \hat{\alpha} \) satisfaz a equação de verossimilhança com erro numérico praticamente nulo (\(1.285 \times 10^{-9}\)).
\end{tcolorbox}

\end{document}
