\documentclass[11pt,a4paper]{article}

% Packages for formatting and graphics
\usepackage[utf8]{inputenc}
\usepackage[portuguese]{babel}
\usepackage{geometry}
\usepackage{graphicx}
\usepackage{listings}
\usepackage{xcolor}
\usepackage{amsmath}
\usepackage{amsfonts}
\usepackage{amssymb}
\usepackage{tcolorbox}

% Page geometry
\geometry{margin=2.5cm}

% R code styling
\definecolor{codegreen}{rgb}{0,0.6,0}
\definecolor{codegray}{rgb}{0.5,0.5,0.5}
\definecolor{codepurple}{rgb}{0.58,0,0.82}
\definecolor{backcolour}{rgb}{0.95,0.95,0.92}

\lstdefinestyle{rstyle}{
    backgroundcolor=\color{backcolour},   
    commentstyle=\color{codegreen},
    keywordstyle=\color{magenta},
    numberstyle=\tiny\color{codegray},
    stringstyle=\color{codepurple},
    basicstyle=\ttfamily\footnotesize,
    breakatwhitespace=false,         
    breaklines=true,                 
    captionpos=b,                    
    keepspaces=true,                 
    numbers=none,                    
    showspaces=false,                
    showstringspaces=false,
    showtabs=false,                  
    tabsize=2,
    frame=single,
    rulecolor=\color{blue!30!black}
}

\lstset{style=rstyle}

\begin{document}

\begin{tcolorbox}[colback=blue!5!white,colframe=blue!75!black,title=Intervalos de Confiança - Exercício 8]

Seja \( X \) uma variável aleatória que representa o desvio (em milímetros) entre o diâmetro observado e o diâmetro nominal de uma peça mecânica fabricada industrialmente.  
Considere que \( X \) tem distribuição normal com valor esperado \( \mu \) desconhecido e desvio padrão conhecido e igual a \( \sigma = 1.1 \).  

\begin{enumerate}
    \item Usando \( R \) e fixando a semente em 1089, gere \( m = 1800 \) amostras de dimensão  
    \[    
    n = 12
    \]  
    de uma distribuição normal com valor esperado igual a \( \mu = 0 \) e desvio padrão \( \sigma = 1.1 \) e determine os respectivos intervalos de confiança para \( \mu \) ao nível de confiança \( \gamma = 0.9 \).  
    
    \item Obtenha a proporção de intervalos de confiança gerados em 1. que contêm o valor esperado \( \mu = 0 \).  
\end{enumerate}

Indique o quociente entre o valor obtido em 2. e o nível de confiança \( \gamma \), arredondado a 4 casas decimais.

\end{tcolorbox}

\section*{Código R}

\begin{lstlisting}[language=R]
# Set parameters
set.seed(1089)
m <- 1800  # number of samples
n <- 12    # sample size
mu <- 0    # true mean
sigma <- 1.1  # known standard deviation
gamma <- 0.9  # confidence level

# Calculate alpha and critical value
alpha <- 1 - gamma
z_alpha_2 <- qnorm(1 - alpha/2)

cat("Parametros:\n")
cat("m =", m, "(numero de amostras)\n")
cat("n =", n, "(tamanho de cada amostra)\n")
cat("mu =", mu, "(valor esperado verdadeiro)\n")
cat("sigma =", sigma, "(desvio padrao conhecido)\n")
cat("gamma =", gamma, "(nivel de confianca)\n")
cat("alpha =", alpha, "\n")
cat("z_{alpha/2} =", z_alpha_2, "\n\n")

# Generate m samples and calculate confidence intervals
confidence_intervals <- matrix(0, nrow = m, ncol = 2)  # Lower and upper bounds
contains_mu <- logical(m)  # Track which intervals contain mu

for (i in 1:m) {
  # Generate sample of size n from normal distribution
  sample_data <- rnorm(n, mean = mu, sd = sigma)
  
  # Calculate sample mean
  x_bar <- mean(sample_data)
  
  # Calculate standard error
  se <- sigma / sqrt(n)
  
  # Calculate confidence interval
  margin_error <- z_alpha_2 * se
  lower_bound <- x_bar - margin_error
  upper_bound <- x_bar + margin_error
  
  # Store interval
  confidence_intervals[i, 1] <- lower_bound
  confidence_intervals[i, 2] <- upper_bound
  
  # Check if interval contains true mu
  contains_mu[i] <- (lower_bound <= mu) & (mu <= upper_bound)
}

# Calculate proportion of intervals that contain mu
proportion_containing_mu <- mean(contains_mu)

# Calculate the quotient
quotient <- proportion_containing_mu / gamma

cat("1. Intervalos de confianca gerados: ", m, "\n")
cat("2. Proporcao de intervalos que contem mu = 0:", proportion_containing_mu, "\n")
cat("Quociente (proporcao / gamma):", quotient, "\n")
cat("Quociente arredondado a 4 casas decimais:", round(quotient, 4), "\n")

# Additional statistics
cat("\nEstatisticas adicionais:\n")
cat("Erro padrao teorico:", sigma/sqrt(n), "\n")
cat("Margem de erro teorica:", z_alpha_2 * sigma/sqrt(n), "\n")
cat("Numero de intervalos que contem mu:", sum(contains_mu), "\n")
cat("Numero de intervalos que NAO contem mu:", sum(!contains_mu), "\n")
\end{lstlisting}

\section*{Resultados}

\begin{tcolorbox}[colback=green!5!white,colframe=green!75!black,title=Solução]
Para a simulação de intervalos de confiança com nível de confiança \( \gamma = 0.9 \):

\begin{center}
\begin{tabular}{|l|c|}
\hline
\textbf{Parâmetro/Resultado} & \textbf{Valor} \\
\hline
Número de amostras (\( m \)) & 1800 \\
Tamanho de cada amostra (\( n \)) & 12 \\
Nível de confiança (\( \gamma \)) & 0.9 \\
\( z_{\alpha/2} \) & 1.644854 \\
\hline
Erro padrão teórico & 0.3175426 \\
Margem de erro teórica & 0.5223112 \\
\hline
1. Intervalos de confiança gerados & 1800 \\
2. Proporção que contém \( \mu = 0 \) & 0.8938889 \\
\hline
\textbf{Quociente (proporção / \( \gamma \))} & \textbf{0.9932099} \\
\hline
\end{tabular}
\end{center}

\vspace{0.5cm}
\textbf{Resposta final:} O quociente entre a proporção obtida em 2. e o nível de confiança \( \gamma \), arredondado a 4 casas decimais, é \boxed{0.9932}.

\vspace{0.3cm}
\textit{Observação:} Dos 1800 intervalos gerados, 1609 contêm o verdadeiro valor de \( \mu = 0 \) e 191 não contêm. O quociente próximo de 1 confirma a validade teórica dos intervalos de confiança.
\end{tcolorbox}

\end{document}
