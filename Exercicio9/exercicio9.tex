\documentclass[11pt,a4paper]{article}

% Packages for formatting and graphics
\usepackage[utf8]{inputenc}
\usepackage[portuguese]{babel}
\usepackage{geometry}
\usepackage{graphicx}
\usepackage{listings}
\usepackage{xcolor}
\usepackage{amsmath}
\usepackage{amsfonts}
\usepackage{amssymb}
\usepackage{tcolorbox}

% Page geometry
\geometry{margin=2.5cm}

% R code styling
\definecolor{codegreen}{rgb}{0,0.6,0}
\definecolor{codegray}{rgb}{0.5,0.5,0.5}
\definecolor{codepurple}{rgb}{0.58,0,0.82}
\definecolor{backcolour}{rgb}{0.95,0.95,0.92}

\lstdefinestyle{rstyle}{
    backgroundcolor=\color{backcolour},   
    commentstyle=\color{codegreen},
    keywordstyle=\color{magenta},
    numberstyle=\tiny\color{codegray},
    stringstyle=\color{codepurple},
    basicstyle=\ttfamily\footnotesize,
    breakatwhitespace=false,         
    breaklines=true,                 
    captionpos=b,                    
    keepspaces=true,                 
    numbers=none,                    
    showspaces=false,                
    showstringspaces=false,
    showtabs=false,                  
    tabsize=2,
    frame=single,
    rulecolor=\color{blue!30!black}
}

\lstset{style=rstyle}

\begin{document}

\begin{tcolorbox}[colback=blue!5!white,colframe=blue!75!black,title=Teste de Hipóteses - Erro Tipo II - Exercício 9]

Considere que $(X_1, \ldots, X_n)$ é uma amostra aleatória de dimensão $n$ de uma população $X$ com distribuição exponencial. Admita que $E(X) = \mu$ é desconhecido e que

\[
T = \frac{2n \bar{X}}{\mu} \sim \chi^2_{(2n)}, \, \forall \mu \in \mathbb{R}^+.
\]

Para testar $H_0 : \mu = \mu_0 = 5$ contra $H_1 : \mu = \mu_1 = 5.4$ pode utilizar-se a estatística de teste $T_0$ obtida de $T$ admitindo que a hipótese $H_0$ é verdadeira, rejeitando-se $H_0$, ao nível de significância $\alpha$, se $T_0 > F^{-1}_{\chi^2_{(2n)}}(1 - \alpha)$.

Fixando a semente em 4382, gere $m = 700$ amostras de dimensão $n = 13$ da distribuição exponencial com valor esperado $\mu_1$. Aplique o teste de hipóteses, ao nível de significância $\alpha = 0.06$, a cada uma das $m = 700$ amostras geradas e calcule uma estimativa, $\hat{\beta}$, da probabilidade teórica do erro de $2º$ espécie, $\beta$.

Obtenha o quociente de $\hat{\beta}$ e $\beta$ e indique o resultado arredondado a 4 casas decimais.

\end{tcolorbox}

\section*{Código R}

\begin{lstlisting}[language=R]
# Set parameters
set.seed(4382)
m <- 700      # number of samples
n <- 13       # sample size
mu_0 <- 5     # null hypothesis mean
mu_1 <- 5.4   # alternative hypothesis mean
alpha <- 0.06 # significance level

# Calculate critical value for chi-square test
# We reject H0 if T0 > chi2_critical
df <- 2 * n
chi2_critical <- qchisq(1 - alpha, df)

cat("Parametros do teste:\n")
cat("n =", n, "\n")
cat("m =", m, "\n")
cat("mu_0 =", mu_0, "\n")
cat("mu_1 =", mu_1, "\n")
cat("alpha =", alpha, "\n")
cat("Graus de liberdade:", df, "\n")
cat("Valor critico chi-quadrado:", chi2_critical, "\n\n")

# Generate m samples from exponential distribution with mean mu_1
# and apply hypothesis test to each sample
rejections <- 0  # Count how many times we reject H0

for (i in 1:m) {
  # Generate sample from exponential with mean mu_1
  # Note: rexp uses rate parameter = 1/mean
  sample_data <- rexp(n, rate = 1/mu_1)
  
  # Calculate sample mean
  x_bar <- mean(sample_data)
  
  # Calculate test statistic T0 under H0 (assuming mu = mu_0)
  T0 <- (2 * n * x_bar) / mu_0
  
  # Test: reject H0 if T0 > critical value
  if (T0 > chi2_critical) {
    rejections <- rejections + 1
  }
}

# Estimate beta (probability of Type II error)
# Beta = P(not reject H0 | H1 is true)
beta_hat <- (m - rejections) / m

cat("Resultados da simulacao:\n")
cat("Numero de rejeicoes de H0:", rejections, "\n")
cat("Numero de nao rejeicoes de H0:", m - rejections, "\n")
cat("Estimativa de beta (beta_hat):", beta_hat, "\n")

# Calculate theoretical beta
# When H1 is true (mu = mu_1), the test statistic follows:
# T = (2*n*X_bar)/mu_0, where X_bar ~ Exp(mu_1)
# We need P(T <= chi2_critical | mu = mu_1)

# Under H1, T = (2*n*X_bar)/mu_0, where X_bar has mean mu_1
# The distribution of T under H1 is more complex, but we can calculate it
# T = (2*n*X_bar)/mu_0 where X_bar ~ Gamma(n, n/mu_1)
# So T ~ Gamma(n, n*mu_0/mu_1) * (2*n/mu_0) = Gamma(n, 2*n/mu_1)

# Actually, 2*n*X_bar/mu_1 ~ chi2(2n) when samples are from Exp(mu_1)
# So T = (2*n*X_bar/mu_0) = (mu_1/mu_0) * (2*n*X_bar/mu_1) ~ (mu_1/mu_0) * chi2(2n)

# The theoretical beta is P(T <= chi2_critical | H1)
# where T ~ (mu_1/mu_0) * chi2(2n)
# This is equivalent to P(chi2(2n) <= chi2_critical * mu_0/mu_1)

adjusted_critical <- chi2_critical * mu_0 / mu_1
beta_theoretical <- pchisq(adjusted_critical, df)

cat("Beta teorico:", beta_theoretical, "\n")

# Calculate quotient
quotient <- beta_hat / beta_theoretical

cat("Quociente beta_hat / beta:", quotient, "\n")
cat("Quociente arredondado a 4 casas decimais:", round(quotient, 4), "\n")

# Additional information
cat("\nInformacoes adicionais:\n")
cat("Valor critico ajustado:", adjusted_critical, "\n")
cat("Poder do teste (1 - beta):", 1 - beta_theoretical, "\n")
cat("Poder estimado (1 - beta_hat):", 1 - beta_hat, "\n")
\end{lstlisting}

\section*{Resultados}

\begin{tcolorbox}[colback=green!5!white,colframe=green!75!black,title=Solução]
Para o teste de hipóteses sobre a distribuição exponencial com \( H_0: \mu = 5 \) vs \( H_1: \mu = 5.4 \) e \( \alpha = 0.06 \):

\begin{center}
\begin{tabular}{|l|c|}
\hline
\textbf{Parâmetro/Resultado} & \textbf{Valor} \\
\hline
Tamanho da amostra (\( n \)) & 13 \\
Número de simulações (\( m \)) & 700 \\
Nível de significância (\( \alpha \)) & 0.06 \\
Graus de liberdade & 26 \\
Valor crítico \( \chi^2 \) & 38.04354 \\
\hline
Número de rejeições de \( H_0 \) & 79 \\
Número de não rejeições de \( H_0 \) & 621 \\
\hline
Estimativa de \( \beta \) (\( \hat{\beta} \)) & 0.8871429 \\
\( \beta \) teórico & 0.8931416 \\
\hline
\textbf{Quociente \( \hat{\beta} / \beta \)} & \textbf{0.9932835} \\
\hline
\end{tabular}
\end{center}

\vspace{0.5cm}
\textbf{Resposta final:} O quociente de \( \hat{\beta} \) e \( \beta \), arredondado a 4 casas decimais, é \boxed{0.9933}.

\vspace{0.3cm}
\textit{Observações:}
\begin{itemize}
    \item Poder teórico do teste: \( 1 - \beta = 0.1069 \)
    \item Poder estimado do teste: \( 1 - \hat{\beta} = 0.1129 \)
    \item O quociente próximo de 1 confirma a precisão da estimação simulada do erro tipo II.
\end{itemize}
\end{tcolorbox}

\end{document}
