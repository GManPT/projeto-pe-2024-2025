\documentclass[11pt,a4paper]{article}

% Packages for formatting and graphics
\usepackage[utf8]{inputenc}
\usepackage{geometry}
\usepackage{graphicx}
\usepackage{listings}
\usepackage{xcolor}
\usepackage{amsmath}
\usepackage{amsfonts}
\usepackage{amssymb}
\usepackage{tcolorbox}

% Page geometry
\geometry{margin=2.5cm}

% R code styling
\definecolor{codegreen}{rgb}{0,0.6,0}
\definecolor{codegray}{rgb}{0.5,0.5,0.5}
\definecolor{codepurple}{rgb}{0.58,0,0.82}
\definecolor{backcolour}{rgb}{0.95,0.95,0.92}

\lstdefinestyle{rstyle}{
    backgroundcolor=\color{backcolour},   
    commentstyle=\color{codegreen},
    keywordstyle=\color{magenta},
    numberstyle=\tiny\color{codegray},
    stringstyle=\color{codepurple},
    basicstyle=\ttfamily\footnotesize,
    breakatwhitespace=false,         
    breaklines=true,                 
    captionpos=b,                    
    keepspaces=true,                 
    numbers=none,                    
    showspaces=false,                
    showstringspaces=false,
    showtabs=false,                  
    tabsize=2,
    frame=single,
    rulecolor=\color{blue!30!black}
}

\lstset{style=rstyle}

\begin{document}

\begin{tcolorbox}[colback=blue!5!white,colframe=blue!75!black,title=Distribuição de Weibull - Exercício 4]

A distribuição de Weibull é aplicada em muitos campos diferentes, como por exemplo, avaliação da fiabilidade de componentes eletrônicas ou de sistemas mecânicos, indústria farmacêutica ou ainda análise de sobrevivência em medicina.

Uma dada variável aleatória \( X \), contínua, não negativa, tem distribuição de Weibull, de parâmetros \( (\lambda, k) \), se a sua função de densidade de probabilidade for dada por

\[
f_X(x) =
\begin{cases} 
\frac{k}{\lambda} \left( \frac{x}{\lambda} \right)^{k-1} \exp \left[ - \left( \frac{x}{\lambda} \right)^k \right], & x \geq 0 \\
0, & x < 0
\end{cases}
\]

onde \( \lambda \in ]0, +\infty [ \) é o chamado parâmetro de escala e \( k \in ]0, +\infty [ \) é o chamado parâmetro de forma. Pode mostrar-se que

\[
E(X) = \int_0^{+\infty} x f_X(x) \, dx = \lambda \Gamma \left( 1 + \frac{1}{k} \right),
\]

onde \( \Gamma(\cdot) \) é a chamada função Gama. Esta função é definida pelo integral impróprio

\[
\Gamma(x) = \int_0^{+\infty} e^{-t} t^{x - 1} \, dt.
\]

O valor \( \Gamma(x) \) pode ser calculado explicitamente para \( x \) inteiro, obtendo-se

\[
\Gamma(x) = (x - 1)!.
\]

Para \( x \) não inteiro é possível calcular valores aproximados usando integração numérica. A função gamma do R faz precisamente isso, o que permite, dados os valores dos parâmetros, \( \lambda \) e \( k \), calcular aproximadamente \( E(X) \). Por outro lado, também é possível calcular um valor aproximado de \( E(X) \), utilizando o método conhecido como integração de Monte Carlo, que consiste em gerar um número muito elevado de observações de \( X \) e calcular a média aritmética respectiva.

Considere uma variável aleatória \( X \) que representa o tempo de vida de uma certa componente aeronáutica (em milhares de horas). Admite-se que \( X \) tem distribuição de Weibull com parâmetros \( \lambda = 18 \) e \( k = 7 \).

\begin{enumerate}
    \item Calcule o valor esperado de \( X \), usando a função gamma.
    \item Fixando a semente em 2134, gere uma amostra de dimensão 5000 de \( X \), e use-a para calcular um valor aproximado de \( E(X) \).
\end{enumerate}

Indique o valor absoluto da diferença entre os valores obtidos em 1. e 2., arredondado a 4 casas decimais.

\end{tcolorbox}

\section*{Código R}

\begin{lstlisting}[language=R]
# Set parameters for Weibull distribution
lambda <- 18  # scale parameter
k <- 7        # shape parameter

# 1. Calculate expected value using gamma function
# E(X) = lambda * Gamma(1 + 1/k)
theoretical_mean <- lambda * gamma(1 + 1/k)
cat("1. Theoretical expected value using gamma function:", theoretical_mean, "\n")

# 2. Monte Carlo approximation
# Set seed for reproducibility
set.seed(2134)

# Generate 5000 samples from Weibull distribution
# Note: R's rweibull uses (shape, scale) parameterization
sample_size <- 5000
weibull_samples <- rweibull(sample_size, shape = k, scale = lambda)

# Calculate sample mean
monte_carlo_mean <- mean(weibull_samples)
cat("2. Monte Carlo approximation (n=5000):", monte_carlo_mean, "\n")

# Calculate absolute difference
absolute_difference <- abs(theoretical_mean - monte_carlo_mean)
cat("Absolute difference:", absolute_difference, "\n")
cat("Absolute difference (rounded to 4 decimal places):", round(absolute_difference, 4), "\n")
\end{lstlisting}

\section*{Resultados}

\begin{tcolorbox}[colback=green!5!white,colframe=green!75!black,title=Solução]
Para uma distribuição de Weibull com parâmetros $\lambda = 18$ (escala) e $k = 7$ (forma):

\begin{center}
\begin{tabular}{|l|c|}
\hline
\textbf{Método} & \textbf{Valor} \\
\hline
1. Valor esperado teórico & 16.83788 \\
2. Aproximação Monte Carlo (n=5000) & 16.81799 \\
\hline
\textbf{Diferença absoluta} & \textbf{0.0199} \\
\hline
\end{tabular}
\end{center}

\vspace{0.5cm}
\textbf{Resposta final:} O valor absoluto da diferença, arredondado a 4 casas decimais, é \boxed{0.0199}.
\end{tcolorbox}

\end{document}