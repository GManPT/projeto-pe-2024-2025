\documentclass[11pt,a4paper]{article}

% Packages for formatting and graphics
\usepackage[utf8]{inputenc}
\usepackage[portuguese]{babel}
\usepackage{geometry}
\usepackage{graphicx}
\usepackage{listings}
\usepackage{xcolor}
\usepackage{amsmath}
\usepackage{amsfonts}
\usepackage{amssymb}
\usepackage{tcolorbox}
\usepackage{enumitem}

% Page geometry
\geometry{margin=2.5cm}

% R code styling
\definecolor{codegreen}{rgb}{0,0.6,0}
\definecolor{codegray}{rgb}{0.5,0.5,0.5}
\definecolor{codepurple}{rgb}{0.58,0,0.82}
\definecolor{backcolour}{rgb}{0.95,0.95,0.92}

\lstdefinestyle{rstyle}{
    backgroundcolor=\color{backcolour},   
    commentstyle=\color{codegreen},
    keywordstyle=\color{magenta},
    numberstyle=\tiny\color{codegray},
    stringstyle=\color{codepurple},
    basicstyle=\ttfamily\footnotesize,
    breakatwhitespace=false,         
    breaklines=true,                 
    captionpos=b,                    
    keepspaces=true,                 
    numbers=none,                    
    showspaces=false,                
    showstringspaces=false,
    showtabs=false,                  
    tabsize=2,
    frame=single,
    rulecolor=\color{blue!30!black}
}

\lstset{style=rstyle}

\begin{document}

\begin{tcolorbox}[colback=blue!5!white,colframe=blue!75!black,title=Teste de Ajustamento de Rayleigh - Exercício 10]

Uma equipa de engenheiros está a analisar a distribuição da velocidade ($X$, em m/s) do vento dominante em determinado local onde se pretende construir uma aerogare, tendo obtido a seguinte amostra com 200 observações:

\begin{small}
2.3, 2.7, 5.2, 0.7, 2.9, 0.6, 2.6, 2.2, 3.8, 0.5, 4.9, 5.4, 3.7, 0.4, 4, 3.6, 2, 0.8, 2.5, 2.8, 1.7, 3.3, \\
1.5, 0.4, 6.4, 1.5, 6, 2.1, 0.4, 4.6, 3.1, 4.4, 4, 2.1, 5, 3.3, 4.7, 3.4, 4.3, 4.5, 2.3, 0.5, 4.9, 3.5, \\
1.8, 1.9, 2.6, 4.3, 4.6, 5.2, 1.6, 2.8, 2.4, 2.8, 1.8, 3.6, 0.8, 5.1, 1.4, 3.2, 1, 6.3, 3.6, 3.6, 1.8, \\
0.9, 4.6, 2.5, 5.8, 0.6, 3.3, 3.2, 6.6, 2.6, 2.5, 1.5, 4.1, 1.7, 2.1, 1.5, 0.4, 4.8, 0.4, 1.5, 4.2, 3.3, \\
1.2, 8.1, 2.4, 2.8, 2.1, 6.3, 4.2, 1.3, 6, 1.3, 3.7, 2.5, 6.6, 2.7, 1.4, 2, 0.7, 4.3, 3.4, 4.3, 4, 4, \\
0.8, 2.3, 2.5, 5.4, 4.3, 0.5, 3.9, 2.2, 3.4, 1.3, 2.4, 4.7, 2, 1.3, 4.4, 2.9, 2.1, 2.5, 1.6, 2.3, 4.4, \\
1.9, 1.9, 1.7, 2, 4.2, 3.4, 3.9, 4.3, 1.3, 2.9, 2.2, 5.1, 2.3, 1.9, 2.9, 5.2, 3.4, 2.6, 2.4, 3.2, 1.3, \\
3.1, 5.1, 1.4, 4.2, 0.9, 1.3, 2.1, 2.6, 6.2, 1.6, 2.7, 1.7, 2.3, 3.3, 2.8, 1.2, 2.6, 1.5, 2, 2.8, 2.5, 2, \\
1.2, 2.2, 2.6, 2.5, 6, 1.9, 3, 3.8, 1.9, 3.2, 3.1, 1.8, 2.6, 1.9, 3.5, 3.7, 1.8, 2.2, 2, 1.3, 2, 1.1, \\
2.2, 3.1, 2.9, 1.3, 0.2, 3.9
\end{small}

Os membros da equipa conjecturam que $X$ possui distribuição de Rayleigh com parâmetro de escala $\sigma$, i.e., com função de distribuição dada por

\[
F_{0}(x)=1-\exp\biggl(-\frac{x^{2}}{2\sigma^{2}}\biggr),\quad x>0.
\]

Teste $H_{0}:X\sim\text{Rayleigh}(\sigma=2.4)$ contra $H_{1}:X\nsim\text{Rayleigh}(\sigma=2.4)$, procedendo do seguinte modo:

\begin{enumerate}[label=\arabic*.]
    \item Fixe a semente em 5885 e selecione ao acaso e sem reposição uma subamostra de dimensão $n=160$ da amostra original.
    
    \item Divida o suporte da variável aleatória $X$, $\mathbb{R}^{+}$, em $k=5$ classes equiprováveis sob $H_{0}$.
    
    \item Agrupe as observações da subamostra selecionada em 1. nas classes definidas em 2. e obtenha o conjunto de frequências absolutas observadas associadas a essas classes.
    
    \item Recorra às frequências absolutas observadas obtidas em 3. e calcule o valor-p do teste de ajustamento do qui-quadrado para as hipóteses referidas.
\end{enumerate}

Com base neste procedimento, indique qual das cinco decisões abaixo deverá tomar a equipa de engenheiros.

\begin{itemize}[label=$\circ$]
    \item[\textsf{a.}] Rejeitar $H_{0}$ aos n.s. de 5\% e 10\% e não rejeitar $H_{0}$ ao n.s. de 1\%.
    
    \item[\textsf{b.}] Rejeitar $H_{0}$ aos n.s. de 1\%, 5\% e 10\%.
    
    \item[\textsf{c.}] Rejeitar $H_{0}$ ao n.s. de 10\% e não rejeitar $H_{0}$ aos n.s. de 1\% e 5\%.
    
    \item[\textsf{d.}] Teste é inconclusivo.
    
    \item[\textsf{e.}] Não rejeitar $H_{0}$ aos n.s. de 1\%, 5\% e 10\%.
\end{itemize}

\end{tcolorbox}

\section*{Código R}

\begin{lstlisting}[language=R]
# Original data (200 observations)
original_data <- c(2.3, 2.7, 5.2, 0.7, 2.9, 0.6, 2.6, 2.2, 3.8, 0.5, 4.9, 5.4, 3.7, 0.4, 4, 3.6, 2, 0.8, 2.5, 2.8, 1.7, 3.3,
1.5, 0.4, 6.4, 1.5, 6, 2.1, 0.4, 4.6, 3.1, 4.4, 4, 2.1, 5, 3.3, 4.7, 3.4, 4.3, 4.5, 2.3, 0.5, 4.9, 3.5,
1.8, 1.9, 2.6, 4.3, 4.6, 5.2, 1.6, 2.8, 2.4, 2.8, 1.8, 3.6, 0.8, 5.1, 1.4, 3.2, 1, 6.3, 3.6, 3.6, 1.8,
0.9, 4.6, 2.5, 5.8, 0.6, 3.3, 3.2, 6.6, 2.6, 2.5, 1.5, 4.1, 1.7, 2.1, 1.5, 0.4, 4.8, 0.4, 1.5, 4.2, 3.3,
1.2, 8.1, 2.4, 2.8, 2.1, 6.3, 4.2, 1.3, 6, 1.3, 3.7, 2.5, 6.6, 2.7, 1.4, 2, 0.7, 4.3, 3.4, 4.3, 4, 4,
0.8, 2.3, 2.5, 5.4, 4.3, 0.5, 3.9, 2.2, 3.4, 1.3, 2.4, 4.7, 2, 1.3, 4.4, 2.9, 2.1, 2.5, 1.6, 2.3, 4.4,
1.9, 1.9, 1.7, 2, 4.2, 3.4, 3.9, 4.3, 1.3, 2.9, 2.2, 5.1, 2.3, 1.9, 2.9, 5.2, 3.4, 2.6, 2.4, 3.2, 1.3,
3.1, 5.1, 1.4, 4.2, 0.9, 1.3, 2.1, 2.6, 6.2, 1.6, 2.7, 1.7, 2.3, 3.3, 2.8, 1.2, 2.6, 1.5, 2, 2.8, 2.5, 2,
1.2, 2.2, 2.6, 2.5, 6, 1.9, 3, 3.8, 1.9, 3.2, 3.1, 1.8, 2.6, 1.9, 3.5, 3.7, 1.8, 2.2, 2, 1.3, 2, 1.1,
2.2, 3.1, 2.9, 1.3, 0.2, 3.9)

# Parameters
set.seed(5885)
n <- 160  # subsample size
sigma <- 2.4  # Rayleigh scale parameter
k <- 5  # number of classes

cat("Dados originais: ", length(original_data), "observacoes\n")
cat("Parametros do teste:\n")
cat("n =", n, "(tamanho da subamostra)\n")
cat("sigma =", sigma, "(parametro de escala Rayleigh)\n")
cat("k =", k, "(numero de classes)\n\n")

# 1. Select random subsample of size 160 without replacement
subsample_indices <- sample(1:length(original_data), n, replace = FALSE)
subsample <- original_data[subsample_indices]

cat("1. Subamostra selecionada (primeiros 10 valores):", head(subsample, 10), "...\n")

# 2. Define k=5 equiprobable classes under H0
# For Rayleigh distribution with CDF F(x) = 1 - exp(-x^2/(2*sigma^2))
# Quantiles for equiprobable classes: 0.2, 0.4, 0.6, 0.8, 1.0

probabilities <- seq(0.2, 1.0, by = 0.2)
# Rayleigh quantile function: Q(p) = sigma * sqrt(-2 * log(1-p))
class_boundaries <- sigma * sqrt(-2 * log(1 - probabilities))

cat("2. Limites das classes equiprovaveis:\n")
cat("   (0,", round(class_boundaries[1], 3), "]\n")
for (i in 2:length(class_boundaries)) {
  cat("   (", round(class_boundaries[i-1], 3), ",", round(class_boundaries[i], 3), "]\n")
}

# 3. Group observations into classes and get observed frequencies
observed_freq <- numeric(k)

for (i in 1:length(subsample)) {
  x <- subsample[i]
  if (x <= class_boundaries[1]) {
    observed_freq[1] <- observed_freq[1] + 1
  } else if (x <= class_boundaries[2]) {
    observed_freq[2] <- observed_freq[2] + 1
  } else if (x <= class_boundaries[3]) {
    observed_freq[3] <- observed_freq[3] + 1
  } else if (x <= class_boundaries[4]) {
    observed_freq[4] <- observed_freq[4] + 1
  } else {
    observed_freq[5] <- observed_freq[5] + 1
  }
}

cat("\n3. Frequencias absolutas observadas:\n")
for (i in 1:k) {
  cat("   Classe", i, ":", observed_freq[i], "\n")
}

# Expected frequency for each class (equiprobable under H0)
expected_freq <- n / k

cat("\nFrequencia esperada por classe:", expected_freq, "\n")

# 4. Chi-square goodness-of-fit test
chi_square_stat <- sum((observed_freq - expected_freq)^2 / expected_freq)
df <- k - 1  # degrees of freedom
p_value <- 1 - pchisq(chi_square_stat, df)

cat("\n4. Teste qui-quadrado:\n")
cat("Estatistica qui-quadrado:", chi_square_stat, "\n")
cat("Graus de liberdade:", df, "\n")
cat("Valor-p:", p_value, "\n")

# Decision at different significance levels
alpha_levels <- c(0.01, 0.05, 0.10)
decisions <- character(length(alpha_levels))

for (i in 1:length(alpha_levels)) {
  if (p_value < alpha_levels[i]) {
    decisions[i] <- "Rejeitar H0"
  } else {
    decisions[i] <- "Nao rejeitar H0"
  }
}

cat("\nDecisoes:\n")
cat("Nivel 1%:", decisions[1], "\n")
cat("Nivel 5%:", decisions[2], "\n")
cat("Nivel 10%:", decisions[3], "\n")

# Determine which option matches the results
if (all(decisions == "Nao rejeitar H0")) {
  answer <- "e"
  cat("\nResposta: e) Nao rejeitar H0 aos n.s. de 1%, 5% e 10%.\n")
} else if (all(decisions == "Rejeitar H0")) {
  answer <- "b"
  cat("\nResposta: b) Rejeitar H0 aos n.s. de 1%, 5% e 10%.\n")
} else if (decisions[1] == "Nao rejeitar H0" && decisions[2] == "Rejeitar H0" && decisions[3] == "Rejeitar H0") {
  answer <- "a"
  cat("\nResposta: a) Rejeitar H0 aos n.s. de 5% e 10% e nao rejeitar H0 ao n.s. de 1%.\n")
} else if (decisions[1] == "Nao rejeitar H0" && decisions[2] == "Nao rejeitar H0" && decisions[3] == "Rejeitar H0") {
  answer <- "c"
  cat("\nResposta: c) Rejeitar H0 ao n.s. de 10% e nao rejeitar H0 aos n.s. de 1% e 5%.\n")
} else {
  answer <- "d"
  cat("\nResposta: d) Teste e inconclusivo.\n")
}
\end{lstlisting}

\section*{Resultados}

\begin{tcolorbox}[colback=green!5!white,colframe=green!75!black,title=Solução]
Para o teste de ajustamento da distribuição de Rayleigh com \( \sigma = 2.4 \):

\begin{center}
\begin{tabular}{|l|c|}
\hline
\textbf{Parâmetro/Resultado} & \textbf{Valor} \\
\hline
Tamanho da subamostra (\( n \)) & 160 \\
Número de classes (\( k \)) & 5 \\
Frequência esperada por classe & 32 \\
\hline
\multicolumn{2}{|c|}{\textbf{Frequências Observadas por Classe}} \\
\hline
Classe 1: (0, 1.603] & 38 \\
Classe 2: (1.603, 2.426] & 37 \\
Classe 3: (2.426, 3.249] & 33 \\
Classe 4: (3.249, 4.306] & 27 \\
Classe 5: $(4.306, +\infty)$ & 25 \\
\hline
\multicolumn{2}{|c|}{\textbf{Teste Qui-Quadrado}} \\
\hline
Estatística qui-quadrado & 4.25 \\
Graus de liberdade & 4 \\
Valor-p & 0.373228 \\
\hline
\end{tabular}
\end{center}

\begin{center}
\begin{tabular}{|l|c|}
\hline
\multicolumn{2}{|c|}{\textbf{Decisões por Nível de Significância}} \\
\hline
Nível 1\% & Não rejeitar \( H_0 \) \\
Nível 5\% & Não rejeitar \( H_0 \) \\
Nível 10\% & Não rejeitar \( H_0 \) \\
\hline
\end{tabular}
\end{center}

\vspace{0.5cm}
\textbf{Resposta final:} Com base no procedimento descrito, a equipa de engenheiros deverá escolher a opção \boxed{\textsf{e}}.

\vspace{0.3cm}
\textit{Justificação:} O valor-p = 0.373 é superior a todos os níveis de significância testados (1\%, 5\% e 10\%), pelo que não se rejeita \( H_0 \) em nenhum dos casos. Os dados são consistentes com a distribuição de Rayleigh proposta.
\end{tcolorbox}

\end{document}
