\documentclass[11pt,a4paper]{article}

% Packages for formatting and graphics
\usepackage[utf8]{inputenc}
\usepackage[portuguese]{babel}
\usepackage{geometry}
\usepackage{graphicx}
\usepackage{listings}
\usepackage{xcolor}
\usepackage{amsmath}
\usepackage{amsfonts}
\usepackage{amssymb}
\usepackage{tcolorbox}

% Page geometry
\geometry{margin=2.5cm}

% R code styling
\definecolor{codegreen}{rgb}{0,0.6,0}
\definecolor{codegray}{rgb}{0.5,0.5,0.5}
\definecolor{codepurple}{rgb}{0.58,0,0.82}
\definecolor{backcolour}{rgb}{0.95,0.95,0.92}

\lstdefinestyle{rstyle}{
    backgroundcolor=\color{backcolour},   
    commentstyle=\color{codegreen},
    keywordstyle=\color{magenta},
    numberstyle=\tiny\color{codegray},
    stringstyle=\color{codepurple},
    basicstyle=\ttfamily\footnotesize,
    breakatwhitespace=false,         
    breaklines=true,                 
    captionpos=b,                    
    keepspaces=true,                 
    numbers=none,                    
    showspaces=false,                
    showstringspaces=false,
    showtabs=false,                  
    tabsize=2,
    frame=single,
    rulecolor=\color{blue!30!black}
}

\lstset{style=rstyle}

\begin{document}

\begin{tcolorbox}[colback=blue!5!white,colframe=blue!75!black,title=Distribuição de Irwin-Hall - Exercício 6]

Sejam \( X_1, \ldots, X_n \) variáveis aleatórias contínuas independentes e identicamente distribuídas a \( X \sim \text{uniforme}(0, 1) \). Então \( S_n = \sum_{i=1}^n X_i \) possui distribuição de Irwin-Hall e

\[
P(S_n \leq x) = \frac{1}{n!} \sum_{k=0}^{\lfloor x \rfloor} (-1)^k \binom{n}{k} (x - k)^n,
\]

para \( 0 \leq x \leq n \) e onde \(\lfloor x \rfloor\) representa a parte inteira do real \( x \).

\begin{enumerate}
    \item Obtenha o valor exato de \( p_n = P(S_n \leq x) \), para \( n = 12 \) e \( x = 6.25 \).
    
    \item Calcule dois valores aproximados de \( p_n \) recorrendo aos métodos seguintes:
    
    \begin{enumerate}
        \item \textbf{Teorema do limite central} (\( p_{n, \text{TLC}} \))
        
        Recorra ao TLC, apesar de \( n \) ser inferior a 30.
        
        \item \textbf{Simulação} (\( p_{n, \text{sim}} \))
        
        \begin{enumerate}
            \item Fixando a semente em 4469, gere \( m = 100 \) amostras de dimensão \( n = 12 \) da distribuição de \( X \).
            
            \item Calcule um valor simulado de \( S_n \) para cada uma das amostras geradas.
            
            \item Obtenha a proporção de valores simulados de \( S_n \) que não excedem 6.25.
        \end{enumerate}
    \end{enumerate}
    
    \item Determine o desvio absoluto entre o valor exato calculado em 1., \( p_n \), e o valor aproximado obtido em 2a., \( p_{n, \text{TLC}} \).
    
    \item Calcule o desvio absoluto entre \( p_n \) e \( p_{n, \text{sim}} \).
    
    \item Calcule o quociente entre os desvios calculados em 3. e 4. e apresente o resultado arredondado a 4 casas decimais.
\end{enumerate}

\end{tcolorbox}

\section*{Código R}

\begin{lstlisting}[language=R]
# Parameters
n <- 12
x <- 6.25
m <- 100  # number of samples for simulation

# 1. Exact value using Irwin-Hall distribution formula
# P(S_n <= x) = (1/n!) * sum_{k=0}^{floor(x)} (-1)^k * choose(n,k) * (x-k)^n

exact_probability <- function(n, x) {
  floor_x <- floor(x)
  sum_term <- 0
  
  for (k in 0:floor_x) {
    term <- (-1)^k * choose(n, k) * (x - k)^n
    sum_term <- sum_term + term
  }
  
  return(sum_term / factorial(n))
}

p_n <- exact_probability(n, x)
cat("1. Valor exato p_n =", p_n, "\n")

# 2a. Central Limit Theorem approximation
# For X ~ Uniform(0,1): E(X) = 0.5, Var(X) = 1/12
# For S_n: E(S_n) = n * 0.5 = 6, Var(S_n) = n * (1/12) = 1

mean_sn <- n * 0.5
var_sn <- n * (1/12)
sd_sn <- sqrt(var_sn)

# Standardize and use normal approximation
z <- (x - mean_sn) / sd_sn
p_n_tlc <- pnorm(z)
cat("2a. Aproximacao TLC p_n_TLC =", p_n_tlc, "\n")

# 2b. Simulation
set.seed(4469)

# Generate m samples of size n from Uniform(0,1)
sn_values <- numeric(m)

for (i in 1:m) {
  # Generate n uniform random variables and sum them
  sample_x <- runif(n, 0, 1)
  sn_values[i] <- sum(sample_x)
}

# Calculate proportion of S_n values <= 6.25
p_n_sim <- mean(sn_values <= x)
cat("2b. Aproximacao por simulacao p_n_sim =", p_n_sim, "\n")

# 3. Absolute deviation between exact and CLT
desvio_tlc <- abs(p_n - p_n_tlc)
cat("3. Desvio absoluto |p_n - p_n_TLC| =", desvio_tlc, "\n")

# 4. Absolute deviation between exact and simulation
desvio_sim <- abs(p_n - p_n_sim)
cat("4. Desvio absoluto |p_n - p_n_sim| =", desvio_sim, "\n")

# 5. Ratio of deviations
quociente <- desvio_tlc / desvio_sim
cat("5. Quociente dos desvios =", quociente, "\n")
cat("5. Quociente arredondado a 4 casas decimais =", round(quociente, 4), "\n")

# Additional information
cat("\nInformacoes adicionais:\n")
cat("Media de S_n (teorica):", mean_sn, "\n")
cat("Desvio padrao de S_n (teorico):", sd_sn, "\n")
cat("Media das simulacoes de S_n:", mean(sn_values), "\n")
cat("Desvio padrao das simulacoes de S_n:", sd(sn_values), "\n")
\end{lstlisting}

\section*{Resultados}

\begin{tcolorbox}[colback=green!5!white,colframe=green!75!black,title=Solução]
Para a distribuição de Irwin-Hall com \( n = 12 \) e \( x = 6.25 \):

\begin{center}
\begin{tabular}{|l|c|}
\hline
\textbf{Método} & \textbf{Valor} \\
\hline
1. Valor exato \( p_n \) & 0.5975144 \\
2a. Aproximação TLC \( p_{n,\text{TLC}} \) & 0.5987063 \\
2b. Aproximação por simulação \( p_{n,\text{sim}} \) & 0.6 \\
\hline
3. Desvio absoluto \( |p_n - p_{n,\text{TLC}}| \) & 0.001191937 \\
4. Desvio absoluto \( |p_n - p_{n,\text{sim}}| \) & 0.002485612 \\
\hline
\textbf{5. Quociente dos desvios} & \textbf{0.4795348} \\
\hline
\end{tabular}
\end{center}

\vspace{0.5cm}
\textbf{Resposta final:} O quociente entre os desvios calculados em 3. e 4., arredondado a 4 casas decimais, é \boxed{0.4795}.

\vspace{0.3cm}
\textit{Observação:} O Teorema do Limite Central proporcionou uma melhor aproximação que a simulação com apenas 100 amostras, resultando num quociente inferior a 1.
\end{tcolorbox}

\end{document}
